% Adapted from layout by Jake Vanderplas
% https://github.com/jakevdp/website/tree/master/CV
%

\documentclass{article} %{{{--
\usepackage[paper=a4paper,
	    top=2cm,
	    left=1.35cm,
	    width=18.2cm,
	    bottom=2cm,
	    ]{geometry}
            %margin=4cm,
\usepackage[english]{babel}
\selectlanguage{english}
\usepackage{calc}
\usepackage[T1]{fontenc}
\usepackage[utf8]{inputenc}
\usepackage{lmodern}
\usepackage[pdftex,  % needed for pdflatex
  breaklinks=true,  % so long urls are correctly broken across lines
  colorlinks=true,
  urlcolor=myblue]{hyperref}
\usepackage{color}
\usepackage[pdftex]{graphicx}
\usepackage{multicol}
\usepackage{wasysym} % For phone symbol
\usepackage{url}
\usepackage{marvosym}

\def\bf{\bfseries}
\def\sf{\sffamily}
\def\sl{\slshape}
% Semi condensed bold

\definecolor{deep_blue}{rgb}{0,.2,.5}
\definecolor{dark_blue}{rgb}{0,.1,.3}
\definecolor{myblue}{rgb}{.01,0.21,0.71}
\definecolor{gray}{rgb}{.5, .5, .5}


%% This gives us fun enumeration environments. 
\usepackage{enumitem}

%% More layout: Get rid of indenting throughout entire document
\setlength{\parindent}{0in}

%% Reference the last page in the page number
%
\usepackage{fancyhdr,lastpage}
\pagestyle{fancy}
%\pagestyle{empty}      % Uncomment this to get rid of page numbers
\fancyhf{}\renewcommand{\headrulewidth}{0pt}
%\fancyfootoffset{\marginparsep}+\marginparwidth}
\lfoot{
  \hspace{-2\marginparsep}
  \,\hfill \arabic{page} of \protect\pageref*{LastPage}  \hfill \,\\
  \,\hfill {\footnotesize \textcolor{gray}{updated February 2014~~~~~}} \hfill\,
}

\newcommand{\mydate}[1]{{\textcolor{gray}{\footnotesize #1}}}


\newcommand{\makeheading}[1]%
        {%\hspace*{-\marginparsep minus \marginparwidth}%
         %\begin{minipage}[t]{\textwidth+\marginparwidth+\marginparsep}%
         \begin{minipage}[t]{\textwidth}%
                {\Large #1}\\%[-0.5\baselineskip]%
                \vskip 0.2\baselineskip
                 \color{deep_blue}{\rule{\columnwidth}{3pt}}%
         \end{minipage}
	 \vskip 1.\baselineskip plus 2\baselineskip minus 1.\baselineskip
	}

\newlength\sidebarwidth
\setlength\sidebarwidth{3.6cm}

\newcommand{\topic}[3][]%
	 {\pagebreak[2]%
	 \vskip 1.5\baselineskip plus 3\baselineskip minus 0.7\baselineskip
	 \begin{minipage}{\textwidth}
         \phantomsection\addcontentsline{toc}{section}{#1}%
         \nopagebreak\hspace{0in}%
         \nopagebreak\begin{minipage}[t]{\sidebarwidth - .2cm}
         \raggedleft \bf\sf 
	 \color{deep_blue}{\Large #2}
	 \end{minipage}%
	 \hfill
	 \begin{minipage}[t]{\linewidth - \sidebarwidth}
	 \nopagebreak{\color{deep_blue}%
		    \rule{0pt}{\baselineskip}%
		    \rule{\linewidth}{2.5pt}%
		    \llap{\raisebox{.3\baselineskip}{\sf #1}}%
		    \vspace*{.1\baselineskip}%
		    }%
	 #3%
	 \end{minipage}
	 \end{minipage}}

\newcommand{\smalltopic}[2]%
	 {\pagebreak[2]%
	 \vskip 1\baselineskip plus 2\baselineskip minus 0.3\baselineskip
	 \begin{minipage}{\textwidth}
	 %\hspace{-\marginparsep minus \marginparwidth}%
         \phantomsection\addcontentsline{toc}{subsection}{#1}%
         \nopagebreak\hspace{0in}%
         \nopagebreak\begin{minipage}[t]{\sidebarwidth - .2cm}
         \raggedleft \bf\sf %\vskip -0.5\baselineskip
	 \textcolor{dark_blue}{\large #1}%
	 \end{minipage}%
	 \hfill
	 \begin{minipage}[t]{\linewidth - \sidebarwidth}
	 \nopagebreak{%
	    %\vspace{-.7\baselineskip}%
	    \rule{\linewidth}{.5pt}%
	    \vspace{.1\baselineskip}%
	    }%
	    #2
	 \end{minipage}
	 \end{minipage}}

\newcommand{\subtopic}[3][]
	 {\begin{minipage}{\textwidth}
	 \vspace*{.4\baselineskip}
         \nopagebreak\hspace{0in}%
         \nopagebreak\begin{minipage}[t]{\sidebarwidth - .2cm}
	 % Super posh: using semi-bold condensed fonts. Works only with
	 % lmodern
         \raggedleft {\sf\fontseries{sbc}\selectfont #2}
	 %{\small\sl\\[-0.2\baselineskip] #1}
         {\\[-0.2\baselineskip] \textcolor{gray}{\footnotesize #1}}
	 \end{minipage}%
	 \hfill
	 \begin{minipage}[t]{\linewidth - \sidebarwidth}
	 #3%
	 \end{minipage}%
	 \vspace*{.2\baselineskip plus 1\baselineskip minus
	 .2\baselineskip}%
	 \end{minipage}}

\newcommand{\dateonly}[2][]
	 {\begin{minipage}{\textwidth}
	 \vspace*{.4\baselineskip}
         \nopagebreak\hspace{0in}%
         \nopagebreak\begin{minipage}[t]{\sidebarwidth - .2cm}
         \raggedleft {~}
         {\\[-\baselineskip] \textcolor{gray}{\footnotesize #1}}
	 \end{minipage}%
	 \hfill
	 \begin{minipage}[t]{\linewidth - \sidebarwidth}
	 #2%
	 \end{minipage}%
	 \vspace*{.2\baselineskip plus 1\baselineskip minus
	 .2\baselineskip}%
	 \end{minipage}}

\newcommand{\sidenote}[2]
	 {\vspace*{-.2\baselineskip}\begin{minipage}{\textwidth}
         \nopagebreak\hspace{0in}%
         \nopagebreak\begin{minipage}[t]{\sidebarwidth - .2cm}
         \raggedleft {#1}
	 \end{minipage}%
	 \hfill
	 \begin{minipage}[t]{\linewidth - \sidebarwidth}
	 #2%
	 \end{minipage}%
	 \vspace*{.5\baselineskip plus 1\baselineskip minus
	 .2\baselineskip}%
	 \end{minipage}}

% New lists environments 
\newlist{outerlist}{itemize}{1}
\setlist[outerlist]{font=\sffamily\bfseries, label=\textbullet}
\setitemize{topsep=0ex, partopsep=0ex}
\setdescription{font=\normalfont\sffamily\bfseries, itemsep=.5ex,
    parsep=.5ex, leftmargin=3ex}

\newcommand{\blankline}{\quad\pagebreak[2]}

\def\mydot{\textcolor{deep_blue}{\rule{1ex}{1ex}}}

%%%%%%%%%%%%%%%%%%%%%%%%%%%%%%%%%%%%%%%%%%%%%%%%%%%%%%%%%%%%%%%%%%%%%--}}}%
\begin{document}
\makeheading{
\begin{minipage}[B]{0.5\textwidth}
    %\vfill
    \vspace*{-.5\baselineskip}%
    \parbox{10cm}{
	\hskip -0.1cm
	{\textcolor{gray}{\huge{\textsl{\textsf{Curriculum Vitae}}}}}
	\hskip -0.1cm
	\vspace{0.25 cm}
	\\
	{\Huge\bf\sf \color{deep_blue} M\huge \hskip -0.05cm IKKO %
	 \Huge \color{deep_blue} L\huge \hskip -0.05cm EPPÄNEN}
	\\[-.1\baselineskip]
    }
\end{minipage}
\hfill
\begin{minipage}[B]{8cm}
    \raggedleft
    \,\vskip -3em
    \small
        Raastuvankatu 22 C 2 \\
        65100 Vaasa\\
        Finland \\
        {\textcolor{blue}\Mobilefone} \textsf{+358 40 730 8564}\\
	      {\textcolor{blue}\Email} \texttt {mleppan23@gmail.com}%
        \vspace*{-.5\baselineskip}%
\end{minipage}
}

%\begin{center} 
%\begin{minipage}{15cm}
\begin{multicols}{2}
\sloppy

%\textcolor{deep_blue}{\bf\sf Research interests}:
%Cosmology, weak lensing, data mining and automated learning algorithms for
%large astronomical data sets.

\begin{itemize}[leftmargin=2ex, itemsep=0ex]
\item[\mydot]
I currently work as a development engineer at Vacon. Mainly I'm responsible 
for main circuit design of AC drives, focusing on inductive/capacitive components. 
Alongside that I do various other tasks including electromagnetic/thermal simulations 
and PDM/PLM usage.
\item[\mydot]
I am interested in programming (self-taught programmer) and open source. 
Nowadays I spent significant time developing mobile applications. My future goal
is to learn more about game programming. I constantly try to improve my skills and 
learn new things weather it's related to some programming language or an open-source 
scientific computing tool.    

\end{itemize}
\end{multicols}
\vspace*{-1.5em}
\fussy
%\end{minipage}
%\end{center}

%%%%%%%%%%%%%%%%%%%%%%%%%%%%%%%%%%%%%%%%%%%%%%%%%%%%%%%%%%%%%%%%%%%%%%%%%%%
\topic{E \large\hskip -1ex DUCATION}{~}

    \subtopic[2003-2009]{\bf MS}{
        Tampere University of Technology \\
        Electrical Engineering(Electromagnetics)\\
    }

    \subtopic[1999-2002]{\ }{
      Upper Secondary School of Kangasala\\
    }
%%%%%%%%%%%%%%%%%%%%%%%%%%%%%%%%%%%%%%%%%%%%%%%%%%%%%%%%%%%%%%%%%%%%%%%%%%%
\topic{E \large\hskip -1ex XPERIENCE}{~}

\vspace*{-0.5\baselineskip}
\smalltopic{Employment}{}

    \subtopic[2008--Present]{Vacon Plc}{
      R\&D Development Engineer}

    \subtopic[2007--2008]{TUT}{
      Research Assistant\\
      Tampere University of Technology}

\pagebreak
%%%%%%%%%%%%%%%%%%%%%%%%%%%%%%%%%%%%%%%%%%%%%%%%%%%%%%%%%%%%%%%%%%%%%%%%%%%
\topic{C \large\hskip -1ex OMPUTING}{
  I am an active developer and I have made several contributions to the open source
  community. 
  For more information, see my Matlab Central File Exchange and Github profiles:\\
  \href{http://www.mathworks.com/matlabcentral/fileexchange/authors/125461}
  {http://www.mathworks.com/matlabcentral/fileexchange/authors/125461}\\
  \href{https://github.com/MikeL83}{https://github.com/MikeL83}
  \\
  Matlab Central File Exchange pick of the week: \\
  \href{http://blogs.mathworks.com/pick/2013/02/22/coding-challenge-on-input-parsing-result/}
  {http://blogs.mathworks.com/pick/2013/02/22/coding-challenge-on-input-parsing-result/}
  }

%\vspace*{-0.5\baselineskip}
\smalltopic{Skills}{

  \begin{itemize}[leftmargin=0ex, itemsep=0ex, labelindent=-2ex, parsep=.5ex]
    \item[\mydot] Proficient in Matlab, Python, C++, Qt/QML and Javascript. 
    \item[\mydot] Experience with a variety of tools and languages, including
      bash, \LaTeX{}, HTML5, CSS3, PHP, Git, SQLite, jQuery, Fortran, LLVM/Clang.
    \item[\mydot] Operating systems: Linux (Ubuntu), Windows XP/7/8 and Sailfish OS. 
  \end{itemize}

}

\smalltopic{Scientific simulation software}{}

   \subtopic[]{Ansys Maxwell}{
     I have used Maxwell electromagnetic field simulation software
     many years for various engineering tasks.}

   \subtopic[]{Ansys Icepak}{
     I have now used Icepak for a few years. I have done several different 
     kind of thermal simulations for the AC drives. In addtion I have participated 
     in several trainings and seminars related to Icepak/thermal engineering.  
   }

   \subtopic[]{Ansys Simplorer}{
     I have used Simplorer for the AC drives simulation.
   }

   \subtopic[]{Ansys Q3D}{
     I have some experience with Q3D related to busbar simulations.  
   }

%%%%%%%%%%%%%%%%%%%%%%%%%%%%%%%%%%%%%%%%%%%%%%%%%%%%%%%%%%%%%%%%%%%%%%%%%%%
\topic{C \large\hskip -1ex OURSES}{~}

    \subtopic[2011]{ECPE}{
        ECPE/Cluster Tutorial: Part 1 - Thermal Engineering of Power Electronic Systems (Thermal Design and           
        Verification) - Erlangen, Germany\\
    }

    \subtopic[2011]{ECPE}{
      ECPE/Cluster Tutorial: Part 2 - Thermal Engineering of Power Electronic Systems (Thermal Management and
      Reliability) - Erlangen, Germany\\
    }
    
    \subtopic[2014]{XSPIS}{
      Cross platform application development using Qt framework\\
    }
    
    \subtopic[2014]{XSPIS}{
       Become a Developer for Sailfish Operating System: An Introduction \\
    }
    
\pagebreak
%%%%%%%%%%%%%%%%%%%%%%%%%%%%%%%%%%%%%%%%%%%%%%%%%%%%%%%%%%%%%%%%%%%%%%%%%%%
\topic{P \large\hskip -1ex ERSONAL DETAILS}{~}

    \subtopic[]{\ }{
        Place of Birth: Loviisa, Finland\\
        Nationality: Finnish\\
        Military Service: July 2002 -- July 2003\\
    }

%%%%%%%%%%%%%%%%%%%%%%%%%%%%%%%%%%%%%%%%%%%%%%%%%%%%%%%%%%%%%%%%%%%%%%%%%%%
\topic{H \large\hskip -1ex OBBIES}{~}

    \subtopic[]{\ }{
        Mountain biking \\
        Indoor rowing \\
        Programming\\
        Gym \\
        Badminton
    }
%%%%%%%%%%%%%%%%%%%%%%%%%%%%%%%%%%%%%%%%%%%%%%%%%%%%%%%%%%%%%%%%%%%%%%%%%%%


\end{document}
